\documentclass[]{book}
\usepackage{lmodern}
\usepackage{amssymb,amsmath}
\usepackage{ifxetex,ifluatex}
\usepackage{fixltx2e} % provides \textsubscript
\ifnum 0\ifxetex 1\fi\ifluatex 1\fi=0 % if pdftex
  \usepackage[T1]{fontenc}
  \usepackage[utf8]{inputenc}
\else % if luatex or xelatex
  \ifxetex
    \usepackage{mathspec}
  \else
    \usepackage{fontspec}
  \fi
  \defaultfontfeatures{Ligatures=TeX,Scale=MatchLowercase}
\fi
% use upquote if available, for straight quotes in verbatim environments
\IfFileExists{upquote.sty}{\usepackage{upquote}}{}
% use microtype if available
\IfFileExists{microtype.sty}{%
\usepackage{microtype}
\UseMicrotypeSet[protrusion]{basicmath} % disable protrusion for tt fonts
}{}
\usepackage[margin=1in]{geometry}
\usepackage{hyperref}
\hypersetup{unicode=true,
            pdftitle={Data Analytics for the Public Good},
            pdfborder={0 0 0},
            breaklinks=true}
\urlstyle{same}  % don't use monospace font for urls
\usepackage{natbib}
\bibliographystyle{apalike}
\usepackage{longtable,booktabs}
\usepackage{graphicx,grffile}
\makeatletter
\def\maxwidth{\ifdim\Gin@nat@width>\linewidth\linewidth\else\Gin@nat@width\fi}
\def\maxheight{\ifdim\Gin@nat@height>\textheight\textheight\else\Gin@nat@height\fi}
\makeatother
% Scale images if necessary, so that they will not overflow the page
% margins by default, and it is still possible to overwrite the defaults
% using explicit options in \includegraphics[width, height, ...]{}
\setkeys{Gin}{width=\maxwidth,height=\maxheight,keepaspectratio}
\IfFileExists{parskip.sty}{%
\usepackage{parskip}
}{% else
\setlength{\parindent}{0pt}
\setlength{\parskip}{6pt plus 2pt minus 1pt}
}
\setlength{\emergencystretch}{3em}  % prevent overfull lines
\providecommand{\tightlist}{%
  \setlength{\itemsep}{0pt}\setlength{\parskip}{0pt}}
\setcounter{secnumdepth}{5}
% Redefines (sub)paragraphs to behave more like sections
\ifx\paragraph\undefined\else
\let\oldparagraph\paragraph
\renewcommand{\paragraph}[1]{\oldparagraph{#1}\mbox{}}
\fi
\ifx\subparagraph\undefined\else
\let\oldsubparagraph\subparagraph
\renewcommand{\subparagraph}[1]{\oldsubparagraph{#1}\mbox{}}
\fi

%%% Use protect on footnotes to avoid problems with footnotes in titles
\let\rmarkdownfootnote\footnote%
\def\footnote{\protect\rmarkdownfootnote}

%%% Change title format to be more compact
\usepackage{titling}

% Create subtitle command for use in maketitle
\newcommand{\subtitle}[1]{
  \posttitle{
    \begin{center}\large#1\end{center}
    }
}

\setlength{\droptitle}{-2em}

  \title{Data Analytics for the Public Good}
    \pretitle{\vspace{\droptitle}\centering\huge}
  \posttitle{\par}
    \author{}
    \preauthor{}\postauthor{}
      \predate{\centering\large\emph}
  \postdate{\par}
    \date{Updated January 25, 2019}

\usepackage{booktabs}
\usepackage{amsthm}
\makeatletter
\def\thm@space@setup{%
  \thm@preskip=8pt plus 2pt minus 4pt
  \thm@postskip=\thm@preskip
}
\makeatother

\usepackage{amsthm}
\newtheorem{theorem}{Theorem}[chapter]
\newtheorem{lemma}{Lemma}[chapter]
\theoremstyle{definition}
\newtheorem{definition}{Definition}[chapter]
\newtheorem{corollary}{Corollary}[chapter]
\newtheorem{proposition}{Proposition}[chapter]
\theoremstyle{definition}
\newtheorem{example}{Example}[chapter]
\theoremstyle{definition}
\newtheorem{exercise}{Exercise}[chapter]
\theoremstyle{remark}
\newtheorem*{remark}{Remark}
\newtheorem*{solution}{Solution}
\begin{document}
\maketitle

{
\setcounter{tocdepth}{1}
\tableofcontents
}
\hypertarget{welcome}{%
\chapter*{Welcome}\label{welcome}}
\addcontentsline{toc}{chapter}{Welcome}

Welcome to the course.

\hypertarget{the-big-promise-of-big-data}{%
\chapter{The Big Promise of Big
Data}\label{the-big-promise-of-big-data}}

\emph{William Seeley and Lauren Zajac (Team 1)}

\hypertarget{topic-overview}{%
\section{Topic Overview}\label{topic-overview}}

The readings this week spent a great deal of time unveiling the
potential, power and promise of big data including what McKinsey and
Company referred to as:

``the new frontier for innovation, competition and productivity''.

Big data has the promise to drive changes as profound as the industrial
revolution according to GE and in 2017 the Economist stated that:

``Data is to this century what oil was to the last century: a driver of
growth and change''.

One of the key insights in the readings this week is that information
overload, or too much information, can be as bad as a lack of
information. Industries and companies who can utilize data specialists
and new technology and tools to work with the data will have a
competitive edge.

\hypertarget{chapter-summaries}{%
\section{Chapter Summaries}\label{chapter-summaries}}

\hypertarget{social-physics-ch1-from-ideas-to-action}{%
\subsection{\texorpdfstring{Social Physics \textbf{CH1 From Ideas to
Action}}{Social Physics CH1 From Ideas to Action}}\label{social-physics-ch1-from-ideas-to-action}}

Alex Pentland examined the tremendous potential of the new frontier of
big data during chapter 1, entitled ``From Ideas to Actions'' when he
presented his concept of social physics. He contends that while the
internet makes our lives more connected, they also make things go
faster, and that we are ``drowning in information, so much that we don't
know what items to pay attention to, and what to ignore'' (p.~2). This
overwhelming volume and speed of information forms virtual crowds across
the world in minutes, leading to catastrophic events like stock market
crashes and downfall of governments (p.2). He recognizes that people can
no longer be seen as independent decision makers, and that the internet
and social media-fueled interactions must be examined to fully explain
and predict human behavior.\\
Social physics is defined as ``a quantitative social science that
describes reliable, mathematical connections between information and
idea flow\ldots{} and people's behavior'' (p.~4). We are able to
understand how ideas flow between people using social learning, and how
this flow ``shapes norms, productivity, creative output'' (p.~4).
Understanding the information flow and resulting changes in behavior is
critical to understanding social physics. Furthermore, social physics is
made possible through Big Data, which tracks our ``digital bread
crumbs'' of all aspects of our lives and choices (p.~8). In addition to
assisting social scientists with ``reality mining'' or analyzing
patterns within these digital bread crumbs, big data also allows us the
opportunity to view society and its complexity ``many orders of
magnitude over prior social science sets'' (p.~11). Pentland also issues
some warnings on the age of Big Data, and he urges that scholars and
researchers follow strict scientific policies to ensure the protection
of privacy at all costs.

\hypertarget{digital-humanitarians-ch1-rise-of-digital-humanitarianism}{%
\subsection{\texorpdfstring{Digital Humanitarians \textbf{CH1 Rise of
Digital
Humanitarianism}}{Digital Humanitarians CH1 Rise of Digital Humanitarianism}}\label{digital-humanitarians-ch1-rise-of-digital-humanitarianism}}

Patrick Meier dives further into the potential of Big Data, and then
discusses his own personal entry into Digital Humanitarianism in Chapter
1, ``The Rise of Digital Humanitarianism''. He describes his own
personal brush with tragedy when his wife was in Port-au-Prince, Haiti,
conducting research in 2010 when the devastating earthquake hit. Feeling
hopeless and powerless, and struggling with the lack of communication or
information, Patrick and his growing network of friends and contacts and
volunteers began to launch a ``Crisis Response Map'' and became digital
humanitarianisms. They used the power of social networks and big data
information to create, publish and maintain this digital map that
pinpointed the worst areas of the disaster and earthquake impact and
cries for help in a single map. This was made possible with thousands of
volunteers world-wide who devoted hours for many weeks, translating and
posting messages, and the map became the source of information from the
press, emergency response teams, and friends and family members seeking
information on their missing and affected loved ones. Big data and the
internet allowed these thousands of volunteers to do something tangible
to help during the crisis, and has forever changed the way we can
respond to disasters. Meier cautions that Big Data and the overflow of
information and data can also be ``as paralyzing as the absence of
data'' (p.~18) and needs to be managed.

\hypertarget{nyt-the-age-of-big-data}{%
\subsection{\texorpdfstring{NYT \textbf{The Age of Big
Data}}{NYT The Age of Big Data}}\label{nyt-the-age-of-big-data}}

Steve Lohr continues this explanation of the power and promise of Big
Data in his 2012 Article, \emph{``The Age of Big Data''} published in
the \emph{New York Times}. He states that Big Data is a \emph{``new
economic asset, like currency or gold''}. Big Data has given rise to a
fast growing career field of data consultants who are available and
adept at analyzing the data to help industries ``make decisions, trim
costs and lift sales'' using Big Data (p.2). And this Big Data is
growing, as it ``more than doubles every two years'', and creates an
``on-line fishbowl\ldots{}into the real time behavior of huge numbers of
people'' (p.~6). Lohr goes on to provide a number of illustrations on
how pervasive the use of Big Data is by presenting a number of examples
in all industries, including business, sports, governments, and
academics. Big Data has also fueled the growth of new computer
technologies to harness and analyze the data, including artificial
intelligence, natural language processing, pattern recognition and
machine learning. Lohr also mentions that there is a feedback loop and
that additional data actually helps make these tools better. Finally,
Lohr also cautions that there are some drawbacks to the data, including
false discoveries, all of the data makes as it is hard to focus on what
is meaningful, biased fact finding, where people look for data that
support their theory, and limits to statistical and mathematical
modeling. But drawbacks aside, Big Data has caused a revolution and
paradigm shift. People are no longer relying on intuition or feelings in
their industries, but are rather focused on the data and analysis.

\hypertarget{key-take-aways-for-yellowdig}{%
\section{Key Take-Aways (for
Yellowdig)}\label{key-take-aways-for-yellowdig}}

\href{https://voicethread.com/myvoice/thread/11963613/71077645/66681745}{VIDEO}

\hypertarget{discussion-questions}{%
\subsection{Discussion Questions}\label{discussion-questions}}

\begin{enumerate}
\def\labelenumi{\arabic{enumi}.}
\item
  Meier cautions that Big Data and the ``overflow of information and
  data'' can also be ``as paralyzing as the absence of data'' (p.~18).
  Describe a time when you encountered ``too much of a good thing
  (data)'' and what were some strategies you used to overcome the
  problem?
\item
  In his article, ``The age of Big Data'', Lohr mentions that
  enthusiasts say that the Big Data has the potential to be ``humanity's
  dashboard'' with numerous helpful and positive uses, while critics
  argue that it is just ``Big Brother'' invading people's privacy. What
  is your feeling on Big Data?
\item
  Meier talks about the tangible results (Digital Humanitarianism!) That
  they found from the use of big data, is there a time that you used
  data and had immediate results?
\end{enumerate}

\hypertarget{references}{%
\section{References}\label{references}}

\begin{enumerate}
\def\labelenumi{\arabic{enumi}.}
\tightlist
\item
  Pentland, A. (2015). \emph{Social Physics: How social networks can
  make us smarter.} Penguin. \textbf{CH1 From Ideas to Action}
\item
  Meier, P. (2015). \emph{Digital humanitarians: how big data is
  changing the face of humanitarian response.} Routledge. \textbf{CH1
  Rise of Digital Humanitarianism}
\item
  The Age of Big Data: New York Times
  \href{https://www.nytimes.com/2012/02/12/sunday-review/big-datas-impact-in-the-world.html}{LINK}
\end{enumerate}

\hypertarget{information-blindness}{%
\chapter{Information Blindness"}\label{information-blindness}}

Team 2 - Matthew Simon and Carlos Lopez

\hypertarget{the-challenge-of-big-data-information-blindness}{%
\chapter{The Challenge of Big Data: Information
Blindness}\label{the-challenge-of-big-data-information-blindness}}

\hypertarget{topic-overview-1}{%
\section{Topic Overview}\label{topic-overview-1}}

In this module's reading the three authors pose challenges to the idea
of ``big data'' and its actual usefulness to those that intend to use
it. Patrick Meier defines big data as high volume, velocity and variety
(Page 28). He uses the idea of social media to make this point. Overall,
the theme between these three readings are how people are blind to the
data they are collecting, its usefulness and how the user is analyzing
the data they are receiving.

Let's take measuring impact for example. Non-profit and government
agencies are always trying to demonstrate how they are measuring impact
and what their contribution to society is. For business, this story is a
little less complicated as they are primarily focused on profitability.
However, in social organizations the overarching vision is a little more
complex than measuring profitability. A common theme throughout the
readings discuss clarity in purpose and focus on the outcome. Are we
truly measuring what it is we (the organization) cares about? Are we
actually collecting the data in which we need to in order to be able to
measure what we care about? Or an even larger philosophical question
posed by Gugerty and Karlan, do we (the organization) actually even know
what we care about.

In our ever-increasing technological world, we are bombarded with
immense amount of information. All of the texts outline human's ability
to process large amounts of information. But they also demonstrate the
short comings of humans being able to analyze data or how we react to it
if it is not digestible in a way to be useful or meaningful. Duhig calls
this the shoving everything in the drawer response.

For us, the biggest over-arching theme to these readings is data's
connection to the work an organization is actually doing and how it can
be utilized to amplify work. You can't just be ok with massive data
collection and not using it. Then it is just a waste of resources. Or if
you are attempting to use it and using it in a way where people are
blind to it or overwhelmed by it, then it becomes an even further waste
of resources. You have to be clear about what it is you are striving for
and what it is you want to collect. You cannot collect data and analyze
before you are clear about why you are collecting and what it is you
want to do with it.

Duhig makes excellent points with regards to the education parallels he
draws. Government policies have been pushing big data collection on
students and student achievement for the past 20 years. The reality is
that policymaker's hearts were in the right place, but local schools
were ill-equipped to process this data. Teachers would become
overwhelmed to this data and weren't using it in a way that could be
useful to their students. Teachers first needed to understand what they
were assessing and why they were assessing it. Then they needed to
understand how far into the data they were going. If a teacher gives a
comprehensive assignment on many topics from an English unit and they
only look at the overall grades on tests; they are not going to know
what they need to do in order to better prepare different groups of
students on their individual needs. The data sets of which they were
already collected were too blunt. They needed to understand how students
performed on various questions. The data needed to be disaggregated in a
better format. Further, there needed to be an investment of time and
training in order to better support teachers in utilizing this data. In
addition to receiving information and data, teachers were forced to
engage with it. They did their own analyses, tested hypothesis, tracked
tests and measurements. By engaging directly with the data they were
better able to use it to improve student performance.

One of the biggest best practices from these readings comes from Gugerty
and Karlan. The reflection questions they pose about theory of change
and how to proceed on measuring outcomes is extremely informative. For
example, they write:

``Validating the initial steps in the theory of change is a critical
step before moving on to measuring impact. Consider a program to deliver
child development, health, and nutrition information to expectant
mothers in order to improve prenatal care and early childhood outcomes.
Starting an impact evaluation before knowing if expectant mothers will
actually attend the training and adopt the practices makes little sense.
First establish that there is a basic take-up of the program and that
some immediate behaviors are being adopted. Before starting an impact
evaluation of a program providing savings accounts, determine whether
people will actually open a savings account when offered, and that they
subsequently put money into the account. If not, the savings account
design should be reconsidered.''

\hypertarget{chapter-summaries-1}{%
\section{Chapter Summaries}\label{chapter-summaries-1}}

\textbf{Digital Humanitarians by Patrick Meier (Pages 25-31)} This
section of reading basically looks at the impact of social media as big
data and its applicable uses to disaster relief. They talk about the
immense amount of social media postings and content and how to parse
through it. Not all of the posts are going to be relevant or timely.
However, they do discus an opportunity with using this type of massive
data availability in response to humanitarian efforts. It all about
identifying what you are looking for.

\textbf{Smarter Faster Better by Charles Duhig ( Chapter 8)} This
chapter gives a variety of practical real-life examples of how people
absorb data. From examples in the school system, which were discussed in
the topic overview, to examples about people being able to choose
retirement accounts. He uses all of these examples to show that people
need to be able to absorb and digest data in an effective way in order
to process it and make a decision. He calls the human ability to make
these choices and breakdown data as scaffolding and winnowing. When
people are able to process data effectively it has huge implications for
the impact that it is able to have on business operations and even the
lives of students.

\textbf{Ten Reasons Not to Measure Impact -- and What to Do Instead by
Mary Kay Gugerty and Dean Karlan} This article focuses on organizations
innate want to measure their impact and sometimes being blinded by what
they are collecting. Governments and funders are increasingly calling on
these organizations to demonstrate what it is they are doing and how
those dollars are being used. They layout some of the missteps that
current organizations fall into and what to do alternatively. For
example, they discuss clarifying a theory of change, deciding on what
programs to actually evaluate over others and how to effectively
integrate data collection into current workstreams.

\hypertarget{key-take-aways-for-yellowdig-1}{%
\section{Key Take-Aways (for
Yellowdig)}\label{key-take-aways-for-yellowdig-1}}

\hypertarget{discussion-questions-1}{%
\subsection{Discussion Questions}\label{discussion-questions-1}}

\begin{enumerate}
\def\labelenumi{\arabic{enumi}.}
\item
  How are you blinded by data in your current organization? Do you feel
  overwhelmed by any data that you receive? What do you do when you
  receive this data?
\item
  Do you feel like you or your organization collect any data that is not
  used for anything? What is the data point? Do you know why it started
  being collected?
\item
  Do you feel that your current data procedures in your organization
  take away time from your work? Do you find data to be informative or
  not in your current practice? Why?
\item
  Disaster affected communities are increasingly becoming ``digital
  communities'' that turn to social media to communicate during
  disasters and to self-organize in response to crises. Do you have your
  own examples of ``digital communities'' related to your organization
  and how does your organization work with them?
\end{enumerate}

\hypertarget{references-1}{%
\section{References}\label{references-1}}

\begin{itemize}
\tightlist
\item
  Duhigg, C. (2016). \emph{Smarter faster better: The secrets of being
  productive. Random House.} \textbf{CH8 pp 238-247, 252-267}
\item
  Meier, P. (2015). \emph{Digital humanitarians: how big data is
  changing the face of humanitarian response. Routledge.} \textbf{CH2
  the rise of big crisis data pp 25-31}\\
\item
  Gugerty, M. K., \& Karlan, D. (2018). Ten reasons not to measure
  impact---And what to do instead. Stanf. Soc. Innov. Rev.
\end{itemize}

\hypertarget{challenges-of-organizational-change}{%
\chapter{Challenges of Organizational
Change"}\label{challenges-of-organizational-change}}

Joseph Lynch Marcela Morales \# The Challenges of Big Data:
Organizational Change

\hypertarget{topic-overview-2}{%
\section{Topic Overview}\label{topic-overview-2}}

With over 2.5 Quintillion bytes of data created every day, the greatest
challenge to business is how to use this data to improve businesses
while making it profitable.

\hypertarget{chapter-summaries-2}{%
\section{Chapter Summaries}\label{chapter-summaries-2}}

Desouza, K. C., \& Smith, K. L. (2014). Big data for social innovation
(Links to an external site.)Links to an external site.. Stanford Social
Innovation Review, 2014, 39-43. \textbf{Big Data for social innovation}

The term ``big data'' is used to describe the growing proliferation of
data and our increasing ability to make productive use of it. The
business community has also been a heavy user of big data. Each month
Netflix collects billions of hours of user data to analyze the titles,
genres, time spent viewing, and video color schemes to gauge customer
preferences to continually update their recommendation algorithms and
programming to give the customer the best possible experience. There, a
large chasm exists between the potential of data-driven information and
its actual use in helping solve social problems. Social problems are
often what are called ``wicked'' problems. Not only are they messier
than their technical counterparts, they are also more dynamic and
complex because of the number of stakeholders involved and the numerous
feedback loops among inter-related components. Numerous government
agencies and nonprofits are involved in tackling these problems, with
limited cooperation and data sharing among them. Then there are policy
and regulatory challenges that need to be faced, such as building
data-sharing agreements, ensuring privacy and confidentiality of data,
and creating collaboration protocols among various stakeholders tackling
the same type of problem. There are multiple dimensions to big data,
which are encapsulated in the handy set of seven ``V''s that follow.
Volume: considers the amount of data generated and collected. Velocity:
refers to the speed at which data are analyzed. Variety: indicates the
diversity of the types of data that are collected. Viscosity: measures
the resistance to flow of data. Variability: measures the unpredictable
rate of flow and types. Veracity: measures the biases, noise,
abnormality, and reliability in datasets. Volatility: indicates how long
data are valid and should be stored. Barriers creating and using big
data include the storage of big data in proprietary systems, the
regulation on data capture, storage, and curating for accountability,
unreliability of data, and the unintended consequence of big data usage.
Recommendations: Building global data banks on critical issues Engaging
citizens and citizen science (Citizens can also be enlisted to help
create and analyze these datasets) Build a cadre of data curators and
analysts (We need to equip students and analysts with the necessary
skills to curate data so as to create large datasets.)

\textbf{Making advanced analytics work for you} Barton, D., \& Court, D.
(2012). Making advanced analytics work for you. Harvard business review,
90(10), 78-83.

\begin{enumerate}
\def\labelenumi{\arabic{enumi}.}
\tightlist
\item
  Choose the right date by mastering the environment you already have
  and exploring surprising sources of information. Be specific about the
  business problem that needs to be solved or opportunities they hope to
  exploit. Get the right technology and IT infrastructure to help
  integrate siloes information (huge issue in government). It will be a
  continuous flow of information so IT infrastructure that reports in
  ``batches'' will not be helpful.
\item
  Identify the business opportunity and determine how the model can
  improve performance. Use hypothesis-led modeling to generate faster
  outcomes and outcomes that are more broadly understood by managers.
\item
  Make it simple.
\end{enumerate}

\textbf{Despite big investments in data, many companies have not made it
profitable}

Despite big investments in data, many companies have not made it
profitable:
\url{https://www.theregister.co.uk/2017/06/07/go_small_on_big_data/}

Mountains of cash keep pouring into the titans of big data despite the
world's inability to do much of value with their software. Companies
like Cloudera and Hortonworks subsequently arose to help mainstream
enterprises put this otherwise complex software to work. It's been a
lucrative gig, with each company raising hundreds of millions of dollars
and, in turn, generating hundreds of millions of dollars in revenue.
What none of them has managed, however, is profit, and that's cause for
concern. In other words, the money keeps pouring into the big data
companies even as their customers generally struggle to figure out how
to turn those investments into meaningful outcomes. These big data
vendors then have to spend mountains of cash to convince would-be
customers that this time it's different, that this time their investment
will return ``actionable insights'' -- that illusive dream of data
scientists everywhere. Indeed, IDG Research nails it when it finds that
``abundant data by itself solves nothing.'' Companies need to scale back
their ambitions to invest in projects that are more evolutionary than
revolutionary in nature, looking to tweak rather than overhaul existing
operational practices.

\textbf{Why Managers hate agile management}

Why managers hate agile management:
\url{https://www.forbes.com/sites/stevedenning/2015/01/28/more-on-why-managers-hateagile/\#186ce9f010ea}

In the traditional model, there is a top down model where a vision or
product is created and this follows a ``relay race'' through the various
managers, line staff, and sales teams. Each level is assigned a
different aspect of the vision or product to achieve an end result. As
noted in the article ``'' Why Do Managers Hate Agile?'' (Forbes, 2015),
the goal of the traditional model was to ``have semi-skilled
employees\ldots{}perform repetitive activities competently and
efficiently'' and coordinating those efforts so that products could be
produced in large quantities.'' In the Agile model, speed to service or
product is the goal which conflicts with the traditional model by using
the concurrent work of many (including private entities) to enhance the
product. The traditional manager is used to having control of the
outcome of the vision or product and this just does not work in the
Agile model which is causing the ``tension.'' To illustrate the
differences between traditional and Agile, the Apple IPhone is a good
model. If Apple had designed the IPhone using the traditional model,
they would release the IPhone with 40 preset applications that they
believed were best using consumer input. Once released, Apple would add
applications based on consumer demand which would be vetted by
management, created by Apple coders, prioritized for release, tested and
placed on the platform. This process would be slow and the variation
between Apple IPhone would be non-existent. All IPhone would have the
same applications loaded. Consumers could seek out competitors with
different variations of applications that met their needs. In the Agile
method (which Apple uses), Apple created an IPhone with a number of
preset applications, however, they have allowed outside entities to
create applications based on the public demand. As of March 2018, there
were 2.1 million apps available in the Apple App Store. In July of 2008,
there were only 800.
(\url{https://www.lifewire.com/how-many-apps-in-app-store-2000252}) This
Agile approach allows the product to stay relevant to the demands of the
consumer rather than the vision of the company. Instead of convincing
the consumer to buy their product, Apple is giving the consumer what
they want as fast as possible. The Agile method releases or lessens the
control that the traditional Manager used to enjoy for the speed and
variation that a wider population can create. The speed to market on
consumer demand is far beyond what a traditional model can keep up with.
The loss of control and power that the traditional Manager has in their
product or service is tough to swallow and that is why Mangers hate
agile.

\hypertarget{key-take-aways-for-yellowdig-2}{%
\section{Key Take-Aways (for
Yellowdig)}\label{key-take-aways-for-yellowdig-2}}

\url{https://youtu.be/1VFlZ_GM3q8}

\hypertarget{discussion-questions-2}{%
\subsection{Discussion Questions}\label{discussion-questions-2}}

Can data be used to solve social issues deemed ``wicked problems'' since
the infrastructure of non-profit agencies and government do not have the
share data in the same way as business.

How can companies and agencies find a way to use big data? Is there a
good roadmap to success?

If data is the such a key to success, why are the largest data companies
have a problem making profit? Why does the Agile model of business
conflict with traditional methods of management?

Why do you think that big data is so important in public sector yet the
availability is so limited?

\hypertarget{references-2}{%
\section{References}\label{references-2}}

\begin{itemize}
\tightlist
\item
  Desouza, K. C., \& Smith, K. L. (2014). Big data for social
  innovation. Stanford Social Innovation Review, 2014, 39-43.\\
\item
  Barton, D., \& Court, D. (2012). Making advanced analytics work for
  you. Harvard business review, 90(10), 78-83.\\
\item
  Despite big investments in data, many companies have not made it
  profitable:
  \href{https://www.theregister.co.uk/2017/06/07/go_small_on_big_data/}{LINK}\\
\item
  Why managers hate agile management:
  \href{https://www.forbes.com/sites/stevedenning/2015/01/28/more-on-why-managers-hate-agile/\#186ce9f010ea}{LINK}
\end{itemize}

\hypertarget{challenges-of-big-data}{%
\chapter{Challenges of Big Data"}\label{challenges-of-big-data}}

Team 4 - Lindsey Duncan and Justin Stoker

\hypertarget{challenges-of-big-data-ethics-and-privacy}{%
\chapter{Challenges of Big Data: Ethics and
Privacy}\label{challenges-of-big-data-ethics-and-privacy}}

\begin{figure}
\centering
\includegraphics{/data-driven-management-textbook/images/cybersecurity2.png}
\caption{Image}
\end{figure}

\hypertarget{topic-overview-3}{%
\section{Topic Overview}\label{topic-overview-3}}

According to the Pew Research Center, 95\% of all adults own some form
of a cell phone and as many as 77\% of those are smart phones (Mobile
Fact Sheet, 2018). Verizon has recently announced their intent to
shutdown their 2G and 3G data streams on December 31, 2019, effectively
pushing people to 4G or the emerging 5G technology for mobile data
(Morris, 2018). Smart phones are just one of the many tools that collect
and report anonymous data based upon its user's location, social
activities, financial transactions, browsing history, and information
searches. This data is collected and passed through algorithms such as
Apple's Siri, Google Maps and Google AdWords to help predict a user's
schedule, interests, traffic patterns and delays, shopping habits, and
more. Big Data is being collected all the time and often without the
knowledge of the individual contributors of that data. This section
discusses the Challenges of Big Data: Ethics and Privacy.

Whether it is the GPS on a cell phone, traffic cameras, license plate
readers, macroscopic infrared imaging, or each other, we are becoming
increasingly aware of the amount of information that is being collected
about our individual lives. To quote the old English saying, ``just
because we can, doesn't mean we should.'' This is an example where
technology is outpacing policy makers -- where policy makers are often
just as oblivious to the what's happening as everyone else.

It is important to discuss the ethics and privacy concerns that come
about from the collection of all our individual data. While the vast
majority of Americans have no concern and claim to have nothing to hide
about their life's data, many are worried about the eventual public
access and public use of that data. ``There are 3 Big Data concerns that
should keep people up at night: Data Privacy, Data Security and Data
Discrimination'' (Marr, 2018). Questions that people are likely to ask
include:

\begin{itemize}
\item
  Is the information truly anonymous or can it be tracked back to me?
\item
  Can my information be used against me?
\item
  Is my data going to be used for corporate enrichment or political
  battles?
\item
  Would my data contribute to racial or other discriminatory profiling
  by government or law enforcement?
\end{itemize}

While the data collection, in general, benefits everyone by helping with
traffic or travel time prediction, allowing your phone to store hours of
your favorite stores, sports scores or news from your favorite teams,
people have begun to express concern about the use or the public
exposure of their personal data for reasons not in the public good. The
book Social Physics establishes the social nature of individuals and
groups and makes the point how information is passes through those
social networks. Currently, social media is used extensively for
everything from sharing personal updates, to business marketing, to news
and press releases. Even in a book that argues the virtues of the
sharing of ideas through social networks, Social Physics acknowledges,
``Maintaining protection of personal privacy and freedom is critical to
the success of any society'' (Pentland, 2015 p.~17).

\hypertarget{chapter-summaries-3}{%
\section{Chapter Summaries}\label{chapter-summaries-3}}

\hypertarget{data-that-turned-the-world-upside-down}{%
\subsection{Data that turned the world upside
down}\label{data-that-turned-the-world-upside-down}}

This article is about an individual researcher named Michal Kosinski and
a Big Data company called Cambridge Analytica. Kosinski's research in
the field of psychometrics (measuring psychological traits) led to the
development of algorithms associating a person's Facebook likes to the
OCEAN (openness, conscientiousness, extroverted, agreeableness,
neuroticism) personality instrument. Kosinski found a person's digital
footprint to be extremely predictive of not only personality, but also
other preferences. Though Kosinski was positive about the uses of his
research, he worried about the potential ramifications. ``What would
happen, wondered Kosinski, if someone abused his people search engine to
manipulate people? He began to add warnings to most of his scientific
work. His approach, he warned, `could pose a threat to an individual's
well-being, freedom, or even life'\,'' (Grassegger, 2017).

Kosinski was concerned when he discovered the work of Cambridge
Analytica which has been associated with President Trump's election
campaign and Great Britain's exit from the European Union (Brexit).
Cambridge Analytica was claimed to have profiled all adults in the U.S.,
using the data for very targeted electronic marketing during the 2016
presidential election. Xx Nix, spokesperson for Cambridge Analytica's
marketing strategy, ``Cambridge Analytica buys personal data from a
range of different sources, like land registries, automotive data,
shopping data, bonus cards, club memberships, what magazines you read,
what churches you attend\ldots{} in the U.S. almost all personal data is
for sale''(Grassegger, 2017). The company then matches this data and
aligns with voter information and the personality profile to identify
the target market.

\hypertarget{eye-in-the-sky-podcast}{%
\subsection{Eye in the Sky Podcast}\label{eye-in-the-sky-podcast}}

Theme is the availability of the data can do a lot of good things, like
solve murders, property crimes, etc. but on the other hand there are
those that call it a ``grotesque violation of privacy'' (Eye in the
Sky). At what point can public data be taken by a person to track down a
cheating spouse? When can it be abused, where do the lines exist?

The Eye in the Sky Podcast details the story of Ross McNutt, a former
military officer that utilized surveillance equipment that continuously
takes pictures every second over the Town of Fallujah in Afghanistan, to
be able to track those that would plant roadside improvised explosive
devices (IED). The surveillance equipment would be attached to the
underside of an aircraft flying well above the town so that people were
nothing more than pixels on a screen. When it was determined that an IED
was planted, it was possible to go back and track the person that set
the device forward to where they hid or met up with others. The method
was effective in tracking down those that would plant the devices.
McNutt later separated from the military and established a private
company called Persistent Surveillance Systems that would do the same
for more domestic towns and cities.

In one example, McNutt demonstrated the use of the technology to track
crime in Juarez, Mexico and ultimately pitched the technology in his
hometown of Dayton, Ohio. Despite reaching out to the American Civil
Liberties Union (ACLU) and local residents, a vocal minority was able to
shut down the proposal over concerns for individual privacy.

\hypertarget{weapons-of-math-destruction-intro-pages-1-13}{%
\subsection{Weapons of Math Destruction -- Intro pages
1-13}\label{weapons-of-math-destruction-intro-pages-1-13}}

The introduction to the Weapons of Math Destruction text recognizes how
the success of Big Data has actually been problematic. Big Data has been
described as more objective than the application of human opinion in
decision making. However, Big Data has also served to reinforce human
bias when it is programmed into the systems used to collect and to
analyze data. Further, O'Neil points out how difficult it is to
challenge the verdict of Big Data because the algorithms and coding are
a closely guarded proprietary secret or are so complex they are
difficult to decipher. ``Like gods, these mathematical models were
opaque, their workings invisible to all but the highest priests in their
domain,: mathematicians and computer scientists'' (O'Neil, 2017, p 3).

The text highlights the problematic use of data, specifically in the
Washington D.C. schools to evaluate teachers. The schools were using
data to evaluate the success of teachers. Those who scored in the lowest
percentiles were separated from employment. This shows how problems
occur with data and algorithms when they are used rather as doctrine
rather than suggestions or indicators. It highlights the story of Sarah
Wysocki who scored well one year and then was fired the next. People
couldn't explain the algorithm and failed to consider suggestive
information that prior year test results on the students may have been
altered by their teachers. Recall the disincentives that occur when what
gets measured gets managed.

The underlying purpose of this text is that there are situations where
Big Data is being misused and it is done by people that don't understand
what they are doing. The author proudly proclaims at the end of the
Introduction, ``Big Data has plenty of evangelists, but I'm not one of
them. This book will focus sharply in the other direction, on the damage
inflicted by WMDs \emph{weapons of math destruction} and the injustice
they perpetuate. Welcome to the dark side of Big Data.'' (O'Neil, 2016,
p.13)

\hypertarget{key-take-aways-for-yellowdig-3}{%
\section{Key Take-Aways (for
Yellowdig)}\label{key-take-aways-for-yellowdig-3}}

\hypertarget{ethics}{%
\subsubsection{Ethics}\label{ethics}}

Users of Big Data should be thoughtful in their approach. As Cathy
O'Neil suggest in Weapons of Math Destruction data can be used for harm
even when intended for good. Programmers and administrators may
inadvertently program personal biases into analytical algorithms. They
should be conscientious in their application of the data, ensuring that
it is not the only means for evaluating success. Success is measured as
the selection of a candidate for a job, the termination of an employee,
the identification of a personal match for dating, the funding of a
program, etc.

Data systems should also be subject to monitoring, evaluation, and
adjustment. If the means for analyzing the data is flawed and hidden
behind a proprietary veil, then the system should be opened up to
scrutiny. If you cannot defend it, you probably shouldn't be doing it.

\hypertarget{privacy}{%
\subsubsection{Privacy}\label{privacy}}

The United States needs something similar to the European Union General
Data Protection Regulation (EUGDPR) to establish policies with teeth to
protect data from breeches and preserve privacy of its citizens. Better
clarity is needed in terms of notification of how businesses are using
data. Notifying customers that video recording is in process or a phone
call is being recorded doesn't necessarily mean that people are
consenting for their images to be used and linked to other forms of data
collection to track personal habits or trends. As Hannes Grassegger and
Michael Krogerus note in The Data That Turned The World Upside Down,
``The company {[}Cambridge Analytica{]} is incorporated in the US, where
laws regarding the release of personal data are more lax than in
European Union countries. Whereas European privacy laws require a person
to `opt in' to a release of data, those in the US permit data to be
released unless a user `opts out.'\,''

The United States needs a data protection standard that encourages
respect of personal data. Penalties for violating data security
according to the EUGDPR can be as much as 4\% of the annual revenue or
€20 million whichever is greater (GDPR, 2019). This penalty is sizable
enough to take data security seriously.

\hypertarget{discussion-questions-3}{%
\subsection{Discussion Questions}\label{discussion-questions-3}}

\begin{itemize}
\item
  All organizations collect and store data in some form or another,
  whether it is for billing, research, marketing, or a host of other
  reasons. As a manager, am I doing what's necessary to protect the data
  that I have from security and privacy breaches?
\item
  Laws currently exist to provide basic security for data protection.
  Should I be doing more, beyond what is necessary, to protect the data
  that I have access to?
\item
  Often data can be collected and then processed through algorithms to
  provide objective performance standards. However data processing is
  only as good as the programmers that prepared the algorithm. Am I
  considering the Human element when drawing conclusions from the data I
  have?
\item
  Do I understand the algorithms or computational methods used to
  interpret the data? Are they accurate? Are they constantly being
  improved to consider additional factors/understandings?
\end{itemize}

\hypertarget{references-3}{%
\section{References}\label{references-3}}

\begin{itemize}
\item
  Eye in the sky:
  \href{https://www.wnycstudios.org/story/eye-sky}{PODCAST}
\item
  Eye in the sky:
  \href{https://www.washingtonpost.com/business/technology/new-surveillance-technology-can-track-everyone-in-an-area-for-several-hours-at-a-time/2014/02/05/82f1556e-876f-11e3-a5bd-844629433ba3_story.html?utm_term=.3be05b5b0d1d}{Washington
  Post}
\item
  European Union: General Data Protection Regulation (2019). Retrieved
  January 17, 2019 from: \href{https://eugdpr.org/the-regulation/}{LINK}
\item
  Feimberg, H. (2016, January 7). FTC Warns Against Use and Misuse of
  Big Data Analytics. Retrieved January 11, 2019 from
  \href{https://www.insightsassociation.org/article/ftc-warns-against-use-and-misuse-big-data-analytics}{LINK}
\item
  Grassegger, H., \& Krogerus, M. (2017, January 28). The Data That
  Turned the World Upside Down. Retrieved January 15, 2019, from
  \href{https://publicpolicy.stanford.edu/news/data-turned-world-upside-down}{LINK}
\item
  Marr, B. (2017, June 15). 3 Massive Big Data Problems Everyone Should
  Know About. Retrieved January 17, 2019 from
  \href{https://www.forbes.com/sites/bernardmarr/2017/06/15/3-massive-big-data-problems-everyone-should-know-about/\#4eeeb8a96186}{LINK}
\item
  ``Mobile Fact Sheet.'' Pew Research Center, 5 Feb.~2018,
  \href{www.pewinternet.org/fact-sheet/mobile/}{LINK}
\item
  Morris, J. (2018, July 2). Verizon 2G and 3G Sunset Starts. Retrieved
  January 17, 2019 from
  \href{https://www.digi.com/blog/verizon-2g-and-3g-sunset-starts/}{LINK}
\item
  O'Neil, C. (2016). Weapons of math destruction: How big data increases
  inequality and threatens democracy. Broadway Books. Introduction pp
  1-13
\item
  Pentland, A. (2015). Social Physics. Penguin Books. p 17
\item
  Header image ``cybersecurity'' By Titima Ongkantong/Shutterstock.com
\end{itemize}

\bibliography{book.bib,packages.bib}


\end{document}
